\documentclass{ctexart}
    \usepackage{mathrsfs}
    \usepackage{multirow}
    \usepackage{graphicx}
    \usepackage{array}
    \usepackage{makecell}
    \usepackage{amsmath}
    \usepackage{booktabs}
    \usepackage{float}
    \usepackage{diagbox}
    \newcommand\mgape[1]{\gape{$\vcenter{\hbox{#1}}$}}
    \newcommand\Ronum[1]{\uppercase\expandafter{\romannumeral #1\relax}}
    \newcommand\ronum[1]{\romannumeral #1\relax}
    \author{钱思天\ 1600011388 No.8}
    \title{实验十五\ 非平衡电桥测量铂电阻的温度系数 \ 实验报告}
    \begin{document}
      \maketitle
      \section{实验数据与处理}
      \subsection{平衡电桥测量结果}
      % Table generated by Excel2LaTeX from sheet 'Sheet1'
% Table generated by Excel2LaTeX from sheet 'Sheet1'
\begin{table}[H]
    \centering
    \caption{不同$R_x$不同$R_1/R_2$(均$E=4.0V$ \& $R_h=0\Omega$)测量结果}
    \resizebox{\textwidth}{!}  
    {
    \begin{tabular}{|c|c|c|c|c|c|c|c|}
        \hline
      \multicolumn{2}{|l|}{\diagbox[dir=NW]{$R_x\  \&\  \frac{R_1}{R_2} $}{测量值}{各待测项}} & $R_0(\Omega )$ & $R_0'(\Omega )$ & $\Delta n$(格)& $R_x(\Omega )$ & $\Delta R_0(\Omega )$ & S \\
      \hline
      $R_{x1}$ & 500/500 & 47.9  & 47.8  & 4.0   & 47.9  & 0.1   & $1.9 \times 10^3$ \\
      \hline
      \multirow{3}[0]{*}{$R_{x2}$} & 50/500 & 3600  & 3575  & 4.0   & 360.0 & 25    & $5.8 \times 10^2$ \\
      \cline{2-8}
      & 500/500 & 360.0 & 361.0 & 4.0   & 360.0 & 1.0   & $1.4 \times 10^3$ \\
      \cline{2-8}                                       
      & 500/500(交换) & 360.0 & 361.0 & 4.0   & 360.0 & 1.0   & $1.4 \times 10^3$ \\
      \hline
      $R_{x3}$ & 500/500 & 4059  & 4005  & 4.0   & 4059.0 & 54    & $3.0\times 10^2$ \\
      \hline  
    \end{tabular}%
    }
    \label{tab:addlabel}%
  \end{table}%
  % Table generated by Excel2LaTeX from sheet 'Sheet1'
\begin{table}[H]
    \centering
    \caption{$R_{x2}$不同测量条件测量结果}
    \resizebox{\textwidth}{!}
    {
      \begin{tabular}{|m{0.365\columnwidth}|c|c|c|c|c|c|c|}
        \hline
        \diagbox[dir=NW,width=0.4\columnwidth]{各测量条件}{测量值}{各待测项}    & $R_0(\Omega )$ & $R_0'(\Omega )$ & $\Delta n$(格) & $R_x(\Omega )$ & $\Delta R_0(\Omega )$ & S \\ \hline
      $E=4.0V$ \& $R_h=0\Omega$ \& $R_1/R_2=500/500$ & 360.0 & 361.0 & 4.0   & 360.0 & 1.0   & $1.4 \times 10^3$ \\ \hline
      $E=2.0V$ \& $R_h=0\Omega$ \& $R_1/R_2=500/500$ & 360.0 & 362.0 & 4.0   & 360.0 & 2.0   & $7.2 \times 10^2$ \\ \hline
      $E=4.0V$ \& $R_h=0\Omega$ \& $R_1/R_2=500/5000$ & 3600  & 3650  & 4.0   & 360.0 & 50.0  & $2.9 \times 10^2$ \\ \hline
      $E=4.0V$ \& $R_h=3.0k\Omega$ \& $R_1/R_2=500/500$ & 360   & 340   & 5.5   & 360.0 & 10.0  & $2.0 \times 10^2$ \\ \hline
      \end{tabular}%
    }
    \label{tab:addlabel}%
  \end{table}%
  
  
      
 关于灵敏度$S$的计算,利用公式
 $$S=\frac{\Delta n}{\Delta R_x /R_x}=\frac{\Delta n}{\Delta R_0 /R_0}$$
 
 可计算出各$S$的实测值,已附于数据表内。

 至于$S$的理论值,根据公式$$S=\frac{S_GE}{R_1+R_2+R_3+R_4+(R_g+R_h)(2+\frac{R_1}{R_x}+\frac{R_0}{R_2})}$$

 将$S_G^{-1}=1.3\times 10^{-6}(A/\mbox{格})$及$R_g=47\Omega$代入,得下二表:
 % Table generated by Excel2LaTeX from sheet 'Sheet1'
\begin{table}[H]
  \centering
  \caption{不同$R_x$不同$R_1/R_2$(均$E=4.0V$ \& $R_h=0\Omega$)$S$理论值计算结果}
    \begin{tabular}{|c|c|c|c|c|c|}
      \hline
    $R_x$ & $R_{x1}$ & \multicolumn{3}{c|}{$R_{x2}$} & $R_{x3}$ \\
    \hline
    $R_1/R_2$ & 500/500 & 50/500 & 500/500 & 500/500(交换) & 500/500 \\
    \hline
    S     & $1.8 \times 10^3$ & $6.2 \times 10^2$ & $1.6 \times 10^3$ & $1.6 \times 10^3$ & $3.2\times 10^2$ \\
  \hline  
  \end{tabular}%
  \label{tab:addlabel}%
\end{table}%

% Table generated by Excel2LaTeX from sheet 'Sheet1'
\begin{table}[H]
  \centering
  \caption{$R_{x2}$不同测量条件$S$计算结果}
    \begin{tabular}{|c|c|c|}
      \hline
    $R_x$ & 条件    & S \\
    \hline
    \multirow{3}[0]{*}{$R_{x2}$} & E=2.0V \& $R_h=0(\Omega )$ \& $R_1/R_2=500/500$ & $8.0 \times 10^2$ \\
    \cline{2-3}
          & E=4.0V \& $R_h=0(\Omega )$ \& $R_1/R_2=500/5000$ & $3.2 \times 10^2$ \\
          \cline{2-3}
          & E=4.0V \& $R_h=3(k\Omega )$ \& $R_1/R_2=500/500$ & $2.2 \times 10^2$ \\
          
          \hline
    \end{tabular}%
  \label{tab:addlabel}%
\end{table}%

 下计算交换桥臂法测得的$R_{x2}$及其不确定度$\sigma_{x2}$:
 
 利用公式$$R=\sqrt{R_{01}\cdot R_{02}}$$
$$\sigma=\sqrt{(\frac{\partial R}{\partial R_{01}})^2\sigma_{R_{01}}^2+(\frac{\partial R}{\partial R_{02}})^2\sigma_{R_{02}}^2+(\delta R)^2}$$
$$(\frac{\partial R}{\partial R_{01}})^2\sigma_{R_{01}}^2=\frac{R_{02}}{4R_{01}}\cdot (\frac{0.1\% \times R_{01}}{\sqrt{3}})^2=0.011$$
$$(\frac{\partial R}{\partial R_{02}})^2\sigma_{R_{02}}^2=\frac{R_{01}}{4R_{02}}\cdot (\frac{0.1\% \times R_{02}}{\sqrt{3}})^2=0.011$$
$$(\delta R_x)^2=(\frac{0.2R_x}{S})^2=0.0026$$

得
$$R_{x2}=\sqrt{R_{01}\cdot R_{02}}=360.0(\Omega)$$
$$\sigma_{x2}=\sqrt{(\frac{\partial R}{\partial R_{01}})^2\sigma_{R_{01}}^2+(\frac{\partial R}{\partial R_{02}})^2\sigma_{R_{02}}^2+(\delta R)^2}=0.2(\Omega)$$
$$R_{x2}\pm\sigma_{x2}=(360.0\pm0.2)\Omega$$
\subsection{其余电阻测量不确定度}
其余电阻均未采用交换桥臂法。因此,其不确定度公式如下:$$\sigma=\sqrt{(\delta R)^2+(\frac{\partial R}{\partial R_{1}})^2\sigma_{R_{1}}^2+(\frac{\partial R}{\partial R_{2}})^2\sigma_{R_{2}}^2+(\frac{\partial R}{\partial R_{0}})^2\sigma_{R_{0}}^2}$$

又:
$$(\delta R)^2=(\frac{0.2R}{S})^2$$
$$(\frac{\partial R}{\partial R_{1}})^2\sigma_{R_{1}}^2=(\frac{R_0}{R_2})^2\frac{(0.1\% R_1)^2}{3}$$
$$(\frac{\partial R}{\partial R_{0}})^2\sigma_{R_{0}}^2=(\frac{R_1}{R_2})^2\frac{(0.1\% R_0)^2}{3}$$
$$(\frac{\partial R}{\partial R_{2}})^2\sigma_{R_{2}}^2=(\frac{R_1R_0}{R_2^2})^2\frac{(0.1\% R_2)^2}{3}$$

得计算结果对应表如下:
% Table generated by Excel2LaTeX from sheet 'Sheet1'
\begin{table}[htbp]
  \centering
  \caption{各测量电阻在给定条件下的不确定度计算值对应表}
  \resizebox{\textwidth}{!}
  {
    \begin{tabular}{|c|c|c|c|}
      \hline
      \diagbox[dir=NW]{实验} {值}{各项} & $R_x$ & 条件    & $\sigma(\Omega)$ \\
          \hline
    \multirow{3}[0]{*}{实验\Ronum{1}} & $R_{x1}$ & E=4.0V \& $R_h=0(\Omega )$ \& $R_1/R_2=500/500$ & 0.05\\
    \cline{2-4}
          & $R_{x2}$ & E=4.0V \& $R_h=0(\Omega )$ \& $R_1/R_2=50/500$ & 0.4 \\
          \cline{2-4}
          & $R_{x3}$ & E=4.0V \& $R_h=0(\Omega )$ \& $R_1/R_2=500/500$ & 5 \\
          \hline
    \multirow{3}[0]{*}{实验\Ronum{2}} & \multirow{3}[0]{*}{$R_{x2}$} & E=2.0V \& $R_h=0(\Omega )$ \& $R_1/R_2=500/500$ & 0.4 \\
    \cline{3-4}
          &       & E=4.0V \& $R_h=0(\Omega )$ \& $R_1/R_2=500/5000$ & 0.4 \\
          \cline{3-4}
          &       & E=4.0V \& $R_h=3(k\Omega )$ \& $R_1/R_2=500/500$ & 0.5 \\
          \hline
    \end{tabular}%
  }
  \label{tab:addlabel}%

\end{table}%
\subsection{S的计算值}
% Table generated by Excel2LaTeX from sheet 'Sheet1'
\begin{table}[H]
  \centering
  \caption{S的理论计算与实际计算值表}
  \resizebox{\textwidth}{!}
  {
    \begin{tabular}{|m{0.13\columnwidth}|c|c|c|c|}
      \hline
      \diagbox[dir=NW]{实验} {值}{各项} & $R_x$ & 条件    & $S_{\mbox{理论}}$ & $S_{\mbox{实际}}$ \\
      \hline
    \multirow{5}[0]{0.15\columnwidth}{实验\Ronum1} & $R_{x1}$ & E=4.0V \& $R_h=0(\Omega )$ \& $R_1/R_2=500/500$ & $1.8 \times 10^3$ & $1.9 \times 10^3$ \\
    \cline{2-5}
          & \multirow{3}[0]{*}{$R_{x2}$} & E=4.0V \& $R_h=0(\Omega )$ \& $R_1/R_2=50/500$ & $6.2 \times 10^2$ & $5.8 \times 10^2$ \\
          \cline{3-5}
          &       & E=4.0V \& $R_h=0(\Omega )$ \& $R_1/R_2=500/500$ & $1.6 \times 10^3$ & $1.4 \times 10^3$ \\
          \cline{3-5}
          &       & E=4.0V \& $R_h=0(\Omega )$ \& $R_1/R_2=500/500$ & $1.6 \times 10^3$ & $1.4 \times 10^3$ \\
          \cline{2-5}
          & $R_{x3}$ & E=4.0V \& $R_h=0(\Omega )$ \& $R_1/R_2=500/500$ & $3.2\times 10^2$ & $3.0\times 10^2$ \\
    \hline
          \multirow{3}[0]{0.15\columnwidth}{实验\Ronum2(略\Ronum1中相同条件)} & \multirow{3}[0]{*}{$R_{x2}$} & E=2.0V \& $R_h=0(\Omega )$ \& $R_1/R_2=500/500$ & $8.0 \times 10^2$ & $7.2 \times 10^2$ \\
          \cline{3-5}
          &       & E=4.0V \& $R_h=0(\Omega )$ \& $R_1/R_2=500/5000$ & $3.2 \times 10^2$ & $2.9 \times 10^2$ \\
          \cline{3-5}
          &       & E=4.0V \& $R_h=3(k\Omega )$ \& $R_1/R_2=500/500$ & $2.2 \times 10^2$ & $2.0 \times 10^2$ \\
    \hline
        \end{tabular}%
  }
  \label{tab:addlabel}%
\end{table}%
\section{思考题}
\subsection{引起非线性误差因素及实验措施}
\paragraph{因素}大致可分为以下几点:
\subparagraph{1}桥臂电阻称不上远大于铂电阻及二桥臂电阻不相等,这就导致了在铂电阻阻值改变的过程中,$U_{out}$不线性输出。
\subparagraph{2}电桥部分存在的接触电阻,导线内阻等电阻,可能会影响到实验。
\subparagraph{3}电流的因素,要保证稳定性。
\subparagraph{4}温度范围还需要在铂电阻随温度线性变化区间内,才能保证线性输出。
\paragraph{措施}针对这些因素,可分别采取如下措施:
\subparagraph{1}使用高精度的大内阻标准电阻,完成测量。
\subparagraph{2}采用三线式接法,并注意导线的选取。
\subparagraph{3}采用稳流源,用万用表检测电流,并利用万用表电压档极高内阻特性,用万用表做电压表测量$U_{out}$。
\subparagraph{4}实验中选取水的冰点到沸点,端点稳定,且位于线性变化区域内。
\subsection{截距问题}
\paragraph{原因}我认为,一来由于温度计的测量精度问题,使得初温的读数存在偏移,同时,$R_0$也可能无法与零点电阻完全相等,从而导致截距不为零。
\paragraph{影响}我认为,对实验结果无影响。本实验中,我们更多的考虑斜率,利用斜率进行计算。当然,如果截距偏离较大,可能零点电阻未能匹配,从而使实验结果不准确的可能性也是存在的。
\section{分析与讨论}
\subsection{比较理论值与实测值}
\paragraph{现象}从数据来看,本次实验所得的实测数据,即便考虑不确定度,也仍然不能使理论值落在区间内。且实测值较理论值更小。

\paragraph{分析}我认为,原因是实验中,由于热敏电阻的改变,我们所采用的线性近似不再准确,即由于电阻增大,流经铂电阻的电流小于$\frac{I_0}{2}$,使得$U_{out}$测量值较小,故而计算时得到的温度系数数值会偏小。



\section{收获与感想}
从前,我知道平衡电桥十分巧妙。

今天,我又感受到了非平衡电桥的神奇。

通过今天的实验,我感受到了不同类型物理量相互转化的思想,从温度计到传感器,无不体现着这一思想的精妙之处。

而且,我也在老师的讲解中,对热力学量的测量精度有了大概的了解,也产生了一定的兴趣。

最重要的是,在本次的实验中,我感受到了,在平衡附近的线性近似的实际应用。这一在题目中常常涉及的方法,终于在单摆之外见到了另一个实际的例子。

在今后的实验课程中,我也会提高自己的实验能力,多想多思考,也去了解一些感觉很普通的事物的不普通的应用。




\end{document} 